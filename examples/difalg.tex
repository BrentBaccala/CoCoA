
\documentclass{article}

\usepackage{vmargin}
\usepackage{times}
\usepackage{graphics}
\usepackage{amsmath, amscd, amsxtra, amsthm}
\usepackage{amssymb}
\usepackage{comment}

\usepackage{supertabular}

\usepackage[cspex,bbgreekl]{mathbbol}

%\setmarginsrb{0.5in}{0.5in}{0.5in}{0.5in}{0.5in}{0.2in}{0pt}{0pt}
\setmargnohfrb{0.5in}{0.75in}{0.5in}{0.5in}

% These settings are for eReader format.  Comment out for regular format.
%\setpapersize{A5}
%\setmargnohfrb{0in}{0in}{0in}{0.1in}

\newcommand{\Lp}[2]{\ensuremath{\text{L}^{#1}(#2)}}
\newcommand{\T}{\ensuremath{\mathbb{T}}}
\newcommand{\ip}[2]{\ensuremath{\langle #1, #2\rangle}}


%\newcommand{\C}{\ensuremath{\mathbb{C}}}
\newcommand{\C}{\ensuremath{{\bf C}}}
\newcommand{\R}{\ensuremath{{\bf R}}}
\newcommand{\GCD}{\ensuremath{\rm{GCD}}}

\pagestyle{empty}

% Get rid of section numbers
\def\thesection{}

% Get rid of page numbers
\def\thepage{}

\begin{document}

\parindent 0pt
\parskip 12pt

\vskip 0.25in

Let's consider the partial differential equation:

$${{\delta^2}\over{\delta x}^2} \Psi = {\delta\over{\delta t}} \Psi$$

The perspective is that of differential algebra.  We fix a
differential field, which is a field equipped with one or more
additional unary operations (derivations), each of which satisfy two
axioms:

\begin{center}
   $D(a+b) = Da+Db$

   $D(ab) = (Da)b + a(Db)$
\end{center}

In this case, we'll use two differential operators ($\delta_x$ and $\delta_t$),
and I'll often write a derivation as either a subscript ($f_x$ is $\delta_x f$)
or an apostrophe.

{\bf Definition} A polynomial $f$ is {\it normal} with respect to a
derivation if it doesn't divide its derivative and {\it special} if it
does.  For an irreducible polynomial, these are the only cases: either
$\GCD(f,f')=1$ (normal) or $\GCD(f,f')=f$ (special).  Ordinary
polynomials are normal.  Bronstein p. 92.

{\bf Lemma} An irreducible normal factor of a polynomial with multiplicity
$p$ and non-zero derivative appears in the polynomial's derivative
with multiplicity $p-1$.  {\bf Proof.}  Write the polynomial as $f^p q$, where
$\GCD(f,q)=1$.  Its derivative is:

$$p f^{p-1} f' q + f^p q' = f^{p-1}(pf'q + fq')$$

$f$ doesn't divide $f'$ (the factor is {\it normal}), and both $f'$
and $q$ are not zero, so $f$ doesn't divide either $pf'q$ or
$(pf'q+fq')$.

{\bf Lemma} An irreducible factor of a polynomial with multiplicity
$p$ and zero derivative appears in the polynomial's derivative with
multiplicity at least $p$, unless the derivative is zero.  Write the
polynomial as $f^p q$, where $\GCD(f,q)=1$.  Its derivative is $f^p
q'$, and while $f$ doesn't factor $q$, it might factor $q'$, or
$q'$ might be zero.

{\bf Lemma} An irreducible normal factor of a polynomial with multiplicity
$p$ appears in the polynomial's derivative with multiplicity at least
$p-1$, unless the derivative is zero.

What field do we use?  Good question!

Let's start with the rational functions in $x$ and $t$ with
complex coefficients: ${\bf C}(x,t)$.  Are there any solutions
to $\delta_x^2\Psi = \delta_t\Psi$ in this differential field?

\begin{comment}

No.  Everything in ${\bf C}(x,t)$ can be written uniquely as a ratio
of polynomials in $x$ and $t$ with GCD 1.  Let's start by considering
a non-trivial denominator and looking at how it behaves under one of
our differential operators, say $\delta_x$:

$$\left({n\over d}\right)' = {{n'd - nd'}\over{d^2}}$$

Treating $n$ and $d$ as polynomials in $x$ with coefficients in
$\C[t]$, we note that $\delta_x$ maps all of these coefficients to
zero, so $d'$ must have lower degree in $x$ than $d$, so $d$ can't
divide $d'$, though they might share a common factor (i.e, $x^2$
doesn't divide $2x$, but they share a common factor of $x$).  Since
$d$ shares no factors with $n$, $nd'$ shares no more than a subfactor
with $d^2$, and $nd'/d^2$, unless it is zero, must have its
denominator $x$-degree greater than $d$'s.  $n'd/d^2 = n'/d$ on the
other hand, can have a denominator $x$-degree no larger than $d$'s, so
applying $\delta_x$ to $n/d$ causes the $x$-degree of the denominator to
increase, unless the denominator didn't involve $x$ at all (the case
where $nd'$ would be zero).

%Now treating $n$ and $d$ as polynomials in $t$ with coefficients in
%$\C[x]$, we easily see that applying $\delta_x$ may change around the
%coefficients, but can't increase the $t$-degree of the denominator (it
%can only decrease).

I was hoping to now show that $\delta_x$ can't increase the $t$-degree
of the denominator, but that's not the case, i.e:

$$\delta_x\left({1\over{xt-1}}\right) = -{t\over{(xt-1)^2}}$$

\end{comment}

\begin{comment}
No.  Let's start by considering how $\delta_x$, when applied to a
polynomial in ${\bf C}[x,t]$, affects its $x$-degree (the highest
power of $x$ in its monomials).

Treating $d$ as a polynomial in $x$ with coefficients in
$\C[t]$, we note that $\delta_x$ maps all of these coefficients to
zero, so $d'$ must have lower degree in $x$ than $d$, so $d$ can't
divide $d'$, though they might share a common factor (i.e, $x^2$
doesn't divide $2x$, but they share a common factor of $x$).
\end{comment}

Everything in ${\bf C}(x,t)$ can be written uniquely as a ratio of
polynomials in $x$ and $t$ with GCD 1, so let's consider an arbitrary
field element $\Psi = n/d$ with $n,d\in\C[x,t]$.  Applying our
differential operators (I now write $\delta_x n$ as $n_x$):

$$\delta_x {n \over d} = {{n_x d - n d_x}\over{d^2}}$$

$$\delta_x^2 {n \over d} = {{n_{xx} d^2 - n d_{xx} d - 2 n_{x}d_{x}d + 2nd_x^2}\over{d^3}}$$

$$\delta_t {n \over d} = {{n_t d - n d_t}\over{d^2}}$$

If $\Psi = {n \over d}$ satisfies $\delta_x^2 \Psi = \delta_t \Psi$, then we must have:

$$n_{xx} d^2 - n d_{xx} d - 2 n_{x}d_{x}d + 2nd_x^2 = n_t d^2 - n d d_t$$

Rearranging:

$$2nd_x^2 - n d_{xx} d = n_t d^2 - n d d_t - n_{xx} d^2 + 2 n_{x}d_{x}d$$

\begin{comment}
Notice that $d$ divides everything on the RHS, so it must also divide the LHS.
We know $\GCD(n,d)=1$, by hypothesis, and we just concluded that
$d$ can't divide $d_x$, so it can't divide the LHS.  Contradiction.
\end{comment}

Consider $f$, a single irreducible factor of $d$,
appearing with multiplicity $p >= 2$, and $f_x \ne 0$, so $\delta_x$ drives
its order down by one.  $f$ appears in $d_x$ with power $p-1$ and in
$d_{xx}$ with power $p-2$, and doesn't appear in $n$ at all, since
$\GCD(d,n)=1$.  Thus, the terms on the LHS have $f$-order exactly
$2p-2$, while all of the terms on the RHS have $f$-order at least
$2p-1$ ($2p$, $2p-1$, $2p$, and $2p-1$, plus possible factors in $n$'s
derivatives, plus another factor in $d_t$ if $f$ does not
involve $t$):

$$2d_x^2 - d_{xx} d = m f^{2p-1}$$

Now let's expand $d=f^pq$:

$$d = f^p q \qquad d_x = f^{p}q_x +pf^{p -1}f_xq$$
$$d_x^2 = f^{2p}q_x^{2} +2pf^{2p -1}f_xqq_x +p^{2}f^{2p -2}f_x^{2}q^{2}$$
$$d_{xx} = f^{p}q_{xx} +2pf^{p -1}f_xq_x +pf^{p -1}f_{xx}q +(p^{2} -p)f^{p -2}f_x^{2}q$$

Substituting into $2d_x^2 - d_{xx} d = m f^{2p-1}$ and cancelling $f^{2p-2}$:

$$-f^{2}qq_{xx} +2f^{2}q_x^{2} +2pff_xqq_x -pff_{xx}q^{2} +(p^{2} +p)f_x^{2}q^{2} = m f$$

$$(p^{2} +p)f_x^{2}q^{2} = m f+f^{2}qq_{xx} -2f^{2}q_x^{2} -2pff_xqq_x +pff_{xx}q^{2}$$

$f$ factors the RHS, but $\GCD(f,q)=1$ and $\GCD(f,f_x)=1$ by
construction, so the LHS must be zero, $p^2 + p = 0$, implying that
$p$ is 0 or -1, both of which contradict the assumption that $p \ge
2$.  So $d$ can have no irreducible factors that involve $x$ and
appear with power greater than unity.

\begin{comment}
The leading term in $2nd_x^2 - n d_{xx} d$ is of power $x^{2r-2}$ and
has coefficient $2r^2 d_r - r(r-1)d_r^2$, so $r^2 + r =0$, and $r$
must be either $0$ or $-1$.  Obviously it can't be $-1$, what about
$0$?  Then $d$ would be a polynomial in $t$ that didn't involve $x$.
\end{comment}

We still have to consider the case of a square-free irreducible factor
involving $x$.  In this case, we can rearrange like this:

$$2nd_x^2 = n_t d^2 - n d d_t - n_{xx} d^2 + n d_{xx} d + 2 n_{x}d_{x}d$$

Note that $d$ divides the RHS, and a purely square-free factor of $d$
involving $x$ would not appear in $d_x$ or $n$, so the LHS must be
zero, which can only occur if either $n=0$ or if $d_x=0$, which
contradicts the assumption that $d$'s factor involves $x$.

To summarize, we've concluded that either $n=0$ or that $d$ must not
involve $x$, and thus be a polynomial in $\C[t]$.  In this case, the
derivatives $d_x$ and $d_{xx}$ vanish from the original equation and it becomes:

$$n_{xx} d^2 = n_t d^2 - n d d_t$$

Cancelling and rearranging:

$$n d_t = n_t d - n_{xx} d$$

$d$ can't divide its own derivative, and $\GCD(n,d)$ is 1 by
hypothesis, so this equation can only hold if $d_t$ is zero, so $d$
must be a constant (LHS), and $n_t = n_{xx}$ (RHS).  This can only
occur if $n$ has the form $ax^2 + 2at + bx +c$, and this, finally, can
be easily seen to satisfy the original differential equation.

So the only solutions to $\delta_x^2 \Psi = \delta_t \Psi$ in
$\C(x,t)$ have the form $a(x^2+2t)+bx+c$.

Can we find a different field that contains more solutions to this PDE?

An extension field is the obvious choice.

It's a basic theorem in differential algebra that differentials extend
in a single unique way into an algebraic extension (Bronstein Theorem
3.2.3), while transcendental extensions require only that the
differential be specified for the primitive element that extends the
field (Bronstein Theorem 3.2.2).

Let's extend $\C(x,t)$ by $r$ (an algebraic) and $z$ (a
transcendental).  I'll set $r^2 = t$, so its differentials are:

$$\delta_x r = 0 \qquad \delta_t r = {1\over{2r}} $$


Since $z$ is transcendental, I can pick (almost) anything for its differentials.  For
reasons that will become apparent later, let's try

$$\delta_x z = -{{x}\over{2t}} z \qquad \delta_t z = {x^2\over{4t^2}} z$$

Now, is there a solution to $\delta_x^2 \Psi = \delta_t \Psi$ in
$\C(x,t,r,z)$?  You bet!  Let's try $\Psi = z/r$.

$$\delta_x \Psi = - {xz\over{2rt}}$$
$$\delta_x^2 \Psi = - {{2zt-x^2z}\over{4rt^2}}$$
$$\delta_t \Psi =  {{x^2z - 2zt}\over{4rt^2}}$$

So, what's the point?  That I can concoct some screwball extension
that solves a PDE?  Actually, $z$ is just an exponential that solves
${{\delta z}\over{\delta y}}=z$ for $y=-x^2/(4t)$, and those weird
differentials are simply the result of applying the chain rule
to compute ${{\delta z}\over{\delta x}}$ and ${{\delta z}\over{\delta t}}$.
Analytically, we would write this solution as:

$$\Psi(x,t) = {e^{-{x^2\over{4t}}}\over{\sqrt{t}}}$$

Later, I'll use differential algebra to show that this is the only
solution (times an arbitrary constant) in $\C(x,t,r,z)$ not in $\C(x,t)$.

The two main extension types I'm interested in are algebraic
extensions (since they've been so heavily studied), and ``holonomic''
extensions of the form ${{\delta z}\over{\delta r}} = f$ for $r$ and
$f$ arbitrary field elements.  This models first order semilinear ODEs.

Then we can start asking questions like, given a finite number of
algebraic and holonomic extensions, can we find solutions to a given
PDE?  This corresponds to asking whether we can solve a PDE by
breaking it down into algebraic functions and ODEs, so would subsume
separation of variables as a special case.

How can we deal with boundary conditions?  Well, once we've found all
solutions to a PDE in a particular extension field, we can then
restrict them to the boundary (say, $t=1$, in the example above), and
ask whether they form a basis for whatever function space (typically
$\text{L}^2$) our boundary condition exists in.  Since exponentials
(as we saw above) are a simple case of the holonomic extension,
Fourier analysis would be subsumed as a special case.

It would be nice to have theorems telling us what conditions are
needed to get a basis set in some extension field, and of course how
to setup that extension and compute the basis elements.

Is there any hope of proving such theorems?  Well, since we're working
with algebra, we've got the machinery of algebraic geometry available
(not something you typically expect in PDE theory).  Reducing modulo
$p$, for example, is definitely in the mix, though this suggestion
does has the flavour of an Army captain musing about tactical nukes
while contemplating an enemy bunker.

If successful, this program should result in a solution technique for
PDEs that would generalize both Fourier analysis and separation of
variables, so I think this is quite promising!


\vfill\eject

What about the intermediate field $\C(x,t,z)$?  Remember

$$z_x = -{{x}\over{2t}} z \qquad z_t = {x^2\over{4t^2}} z$$

{\bf Lemma}  For all $f \in \C[x,t,z], f \notin \C[x,t] \implies f_x \ne 0$.
Consider $f = \sum f_i z^i$.  Then:

$$f_x = \sum \left( f_{ix} - {ix\over{2t}} f_i \right) z^i$$

If $f_x = 0$, then each of these terms must be 0.  The $x$-degree of $f_{ix}$
is one less than the $x$-degree of $f_i$, but the $x$-degree of $xf_i$ is
one greater than that, so either $i$ or $f_i$ is zero.  So all of the $f_i$
are zero except $f_0$, but then $f$ would be in $\C[x,t]$.  QED.

Returning to the original problems, we're once again led to consider the equation:

$$n_{xx} d^2 - n d_{xx} d - 2 n_{x}d_{x}d + 2nd_x^2 = n_t d^2 - n d d_t$$

This time, we'll split $d$ into its normal and special components,
with respect to $\delta_x$:

$$d =d_s d_n$$

This time, we consider an irreducible normal factor $f$ of $d_n$ with
non-zero $x$-derivative, and use the same logic as before to conclude
that $f_x$ is zero, so $d_{nx}$ is zero.  By the previous lemma,
$d_{n}$ must be in $\C[t]$.

Next we attack $d_s$, using Bronstein's Theorem 5.1.2 (p. 130) which
says that $d_s$ has to be a power of $z$.

$$d=z^a f \qquad f \in \C[t]$$

$$d_x = a z^a f \left( - {x\over{2t}} \right)$$

$$d_{xx} = a^2 z^a f \left( - {x\over{2t}} \right)^2 - a z^a f \left( {1\over{2t}} \right)$$

$$d_t = z^a \left( a {x^2\over{4t^2}} f + f_t\right)$$

Our polynomial equation (after cancelling $f$ and $z^{2a}$, and clearing the denominator) becomes:

$$(a^{2} +a)x^{2}fn +4axftn_x +4ft^{2}n_{xx} -4ft^{2}n_t +2aftn +4f_tt^{2}n = 0$$

This implies that $4f_tt^{2}n$ must be a multiple of $f$, and since $f$ is irreducible and normal,
and $\gcd(f,n)=1$, this can only occur if $f=t$.

So now let's consider a denominator of the form $z^p t^a f$, where $f$ has no $t$ factor:

$$(p^{2} +p)x^{2}fn +4pxtfn_x +4t^{2}fn_{xx} -4t^{2}fn_t +4t^{2}f_tn +(2p +4a)tfn = 0$$

Again the minimum power of $f$ appears multiplied by $4f_tt^{2}n$, which is a contradiction
(since there's no $t$ factor in $f$), implying that $f_t$ is zero.

So we've reduced the denominator to the form $z^a$.  Expanding the polynomial equation we get:

$$(a^{2} +a)x^{2}n +4axtn_x +4t^{2}n_{xx} -4t^{2}n_t +2atn = 0$$

Now we look at the powers of $t$ and conclude that $(a^{2}
+a)x^{2}n$ must be a multiple of $t$, implying that the numerator
must include a $t$ factor.  Expanding that out, we obtain the same
multiple as before, leading to the conclusion that $a^2+a$ must be
zero, so $a$ must be 0 or -1.  Since it can't be -1, it must be zero,
and the denominator is therefore trivial.

Now let's look at the numerator, assume that it has the form $\sum n_i z^i$, where the $n_i$
are polynomials in $\C[x,t]$.  Expanding out $n_{xx} - n_t = 0$, we obtain:

$$(i^{2} -i)x^{2}n_i -4ixtn_{ix} +4t^{2}n_{ixx} -4t^{2}n_{it} -2itn_i$$

Looking at the powers of $t$, we conclude that $(i^{2} -i)x^{2}n_i$ must be a multiple
of $t$, so we now try a numerator of the form $\sum n_i t^a z^i$, and again
conclude that $(i^2-i) x^2 n_i$ must be a multiple of $t$, which implies that $i^2-i=0$
and thus $i$ is either zero or one.

Our numerator equation for $i=1$ assumes the form:

$$-2n_x x - n + 2 t n_{xx} = 2 t n_t$$

Assuming that $n$ has the form $\sum a_b t^b$, where the $a_b$ are polynomials in
$\C[x]$, we expand this into:

$$-2\sum x n_{bx} t^b - \sum a_b t^b + 2 \sum a_{b-1xx} t^{b} = 2 \sum b a_b t^b$$

The $b=0$ term becomes $-2xa_{0x} - a_0 = 0$.  Since $a_0$ is a polynomial
in $\C[x]$, this equation implies that each of its coefficients would have
to be at least twice itself, which is impossible, so $a_0=0$.  Likewise,
$a_0=0$ implies that $a_1$'s equation is $-2xa_{1x}-a_1=2a_1$, which is
likewise impossible.  By induction, $a_b=0$ implies that $a_{b+1}=0$,
so all of the $a_b$ are zero and there is no $z$ ($i=1$) term in
the numerator.

The numerator thus reduces to the $i=0$ term, for which $n_{xx}=n_t$
and we have the same solution set as before: $a(x^2+2t)+bx+c$.

\vfill\eject

We should be able to find more solutions in $\C(x,t,r,z)$, where $r^2=t$.

Let's study this field as the fraction field of the ring
$\C(x,t,r)[z]$.  This enables us to cleanly analyse $z$ as a monomial
extension.  Any polynomial in $z$ will have a non-zero $x$-derivative
(previous lemma), and we've already dealt with the denominator cases for
normal factors with $f_x \ne 0$, so we once again consider a special
denominator of the form $z^a$, and a numerator of the form $t^b n +
t^c n_r r$ where both $n$ and $n_r$ are in $\C(x,t)$.  This leads
to a pair of equations:

$$(a^{2} +a)x^{2}t^{b}n +4axt^{b +1}n_{x} +4t^{b +2}n_{xx} -4t^{b +2}n_{t} +(2a -4b)t^{b +1}n = 0$$

$$(a^{2} +a)x^{2}t^{c}n_r +4axt^{c +1}n_{rx} +4t^{c +2}n_{rxx} -4t^{c +2}n_{rt} +(2a -4c -2)t^{c +1}n_r = 0$$

Since $t$ can not factor either $n$ or $n_r$, $a^2+a$ must be zero, again
leading to the conclusion that $z$ does not appear in the denominator.

The numerator analysis, based on $\sum n_i t^a z^i$ is unchanged, leading
again to the conclusion that $i$ is either zero or one.  The two resulting
equations are:

\begin{align*}
n_{xx} = n_t   & \qquad (i=0) \\
-2n_x x - n + 2 t n_{xx} = 2 t n_t  &  \qquad (i=1)
\end{align*}

This time our solution space is $\C(x,t,r)$, however.  Assuming $n$
has the form $n + n_r r$ (remember that no fraction field is needed
for algebraic extensions), we obtain four equations to be solved in $\C(x,t)$:

\begin{align*}
n_{xx} = n_t  & \qquad (i=0) \\
n_r = 2tn_{rxx} -2tn_{rt} & \qquad (i=0; r) \\
-2xn_{x} +2tn_{xx} -2tn_{t} -n = 0 & \qquad (i=1) \\
-2xn_{rx} +2tn_{rxx} -2tn_{rt} -2n_r = 0 & \qquad (i=1; r)
\end{align*}

These four rational equations convert to four polynomial equations:

$$-ndd_{xx} +ndd_t +2nd_x^{2} -2n_xdd_x +n_{xx}d^{2} -n_td^{2} = 0$$
$$-2tndd_{xx} +2tndd_t +4tnd_x^{2} -4tn_xdd_x +2tn_{xx}d^{2} -2tn_td^{2} -nd^{2} = 0$$
$$2xndd_x -2xn_xd^{2} -2tndd_{xx} +2tndd_t +4tnd_x^{2} -4tn_xdd_x +2tn_{xx}d^{2} -2tn_td^{2} -nd^{2} = 0$$
$$2xndd_x -2xn_xd^{2} -2tndd_{xx} +2tndd_t +4tnd_x^{2} -4tn_xdd_x +2tn_{xx}d^{2} -2tn_td^{2} -2nd^{2} = 0$$

The first one we've analysed already; its solutions are $ax^2 + 2at + bx +c$.

Analyzing the remaining three using the techniques already discussed
(irreducible factors) reveal that their denominator's $x$-derivatives
are zero.  Setting $d_x$ and $d_{xx}$ to zero, we obtain:

$$2tnd_t +2tn_{xx}d -2tn_td -nd = 0$$
$$-2xn_xd +2tnd_t +2tn_{xx}d -2tn_td -nd = 0$$
$$-2xn_xd +2tnd_t +2tn_{xx}d -2tn_td -2nd = 0$$

Considering the next one, we see that $2tnd_t$ must be a multiple of $d$, so we're led to
consider $t$ as a factor of $d$, i.e, $d=t^a q$:

$$2nt q_t + 2tn_{xx}q -2tn_tq +(2a-1)nq = 0$$

which requires either $2ntq_t$ to be a multiple of $q$ (impossible), or $a={1\over2}$,
so the second equation has no solution.

The next-to-last equation also requires either $2ntq_t$ to be a multiple of $q$ or $a={1\over2}$, so it's also impossible.

The last equation becomes:

% $$-2xn_xt^aq +2at^anq + 2 t^{a+1} n q_t +2n_{xx}t^{a+1}q -2n_tt^{a+1}q -2nt^aq = 0$$

$$-2xn_xq  + 2 n t q_t +2tn_{xx}q -2tn_tq +(2a-2)nq = 0$$

At first, this appears to require $2ntq_t$ to be a multiple of $q$, but there is another possibility!
If $n$ and $q$ are both constant, then the equation reduces to:

$$(2a-2)nq = 0$$

which can be satisfied if $a=1$, which means $d=tq$.

In short, we've found another solution: $zr/t$, as expected from analysis.
The general solution in $\C(x,t,r,z)$ is:

$$a(x^2+2t)+bx+c+d{zr\over t}$$


\vfill\eject

\section*{Solving Polynomial Differential Equations}

Often we're confronted with an equation like $n_{xx} = n_t$, where $n$
is constrained to be a polynomial.  A fair amount is known about
such equations, but there are still big gaps.

An important result is the Abramov-Petkov\v sek theorem, which proves
the impossibility of algorithmically solving PDEs in a polynomial ring
as a straightforward consequence of the undecidability of Hilbert's
tenth problem.  The proof is based on the close correlation between
solving polynomial differential equations and solving integer Diophantine
equations.  Abramov and Petkov\v sek demonstrated a simple
construction that allows any polynomial equation to be converted
into a PDE with the property than any polynomial solution to the PDE
can be trivially lifted to a solution of the original polynomial.
Thus, any algorithm that could solve arbitrary PDEs in a polynomial
ring could be used to solve arbitrary Diophantine equations, which
is impossible (the MRDP theorem).

However, their proof technique suggests to me that the degree of the
resulting Diophantine equations should be bounded by the order of the
PDE.  PROOF NEEDED.  If true, then this is good news since Sch\"odinger's
equation is second order and Cohen's GTM 239 suggests that second
degree Diophantine equations may be completely solvable.  See
section 6.3 (p. 341) in GTM 239.

The form of the Diophantine equations can be further restricted.
Since there are no cross derivatives of the form $\delta^2/\delta x \delta y$,
there should be no mixed monomials in the Diophantine equations,
so they can be separated into a linear component (solvable using
matrix reduction into Hermite normal form) and a sum of squares.
If the squares all have the same sign, which is likely since
the second derivatives in Schr\" odinger's equation are invariant
under exchange of variables, then the equation is of elliptic
type, meaning that it describes an n-dimensional ellipsoid
in real space $\R^n$, so its integer solutions are bounded
and thus computable.  Different values from the linear
component, however, might lead to an infinite number of
elliptic quadratics with unbounded total size, and it
also seems likely that a single PDE could generate an
infinite number of Diophantine equations.

Systems of partial differential equations have been extensively
studied in the form of D-modules, and algorithms have been developed
to find their polynomial and rational solutions in the case of finite
holonomic rank.  Most of our equations are of infinite rank, however.
The known D-module algorithms for rational solutions depend on finding
an algebraic variety (the singular locus), which is basically the
Zariski closure of the solution's singularities and thus provides
crucial information about what factors can be present in the
denominator.  This seems difficult to generalize into transcendental
extension fields, since the relationship between $x$ and $e^x$, for
example, is not algebraic.  Futuremore, Tsai's Lemma 2.1.5 states that in
the infinite rank case, the singular locus is trivial (it's the entire space).

\vfill\eject

\section*{Operating in K[Z[p]]}

Often we need to operate in a monoid ring like $K[Z[p]]$, i.e, a
polynomial ring whose exponents are polynomials.  For example:

$$x^{2p}+2x^p+1=(x^p+1)^2$$

To order the monomials, an ordering is required on the exponents.  I use a ``high''
ordering where the coefficient of the highest power determines the comparison to zero.
\footnote{\tt http://math.arizona.edu/$\sim$rwilliams/math415A-fall2013/Ordered\_Rings\_and\_Fields.pdf}
Only monoid elements greater than zero are allowed as exponents, so $x^{p-50}$ is
in $K[Z[p]]$, but $x^{-1}$ is not.

We can't use Buchberger's algorithm directly, because it may not terminate since
the ring is not Noetherian.

$$(x^p) \subset (x^{p-1}) \subset (x^{p-2}) \subset \cdots$$

is an ascending chain of ideals that never stabilises.
For the same reason, $K[Z[p]]$ is not a unique factorization domain:

$$x^p = x^{p-10}x^{10} = x^{p-20}x^{20} = \cdots$$

However, $K[Z[p]]$ is a GCD domain.  PROOF NEEDED.

We can compute GCDs (needed for normalisation in the fraction field) using a
simple modification to Buchberger's algorithm.  Introduce new indeterminates
for each combination of lower and upper indeterminates.

$$x^p  \to  a$$
$$x^q  \to  b$$
$$x^{(p+q)} \to ab$$

This produces new polynomials with only integer exponents.  We can now
compute a Gr\"obner basis, which is how CoCoA computes GCDs, then map
the results back into the original ring.

How do we deal with monomials like $x^{p-1}$?  We can't factor it like
$x^px^{-1}$ because $x^{-1}$ isn't in our ring.  Instead, we map
$x^{p-1} \to a$ and then write $x^{p-1}$ as $a$ and $x^p$ as $xa$.
This implies that we can't construct a single derived ring that can
handle all polynomials in $K[Z[p]]$.  Instead, we construct a
derived ring for any particular GCD calculation.

The derived ring is Noetherian, but is not isomorphic to $K[Z[p]]$.
Instead, it is isomorphic to a subring of $K[Z[p]]$ and we can adjust
the construction so that the subring encompasses any finite number of
elements from $K[Z[p]]$.

Theorem.  Given a ring R, if every pair of elements from R can be
embedded in a subring that is a GCD domain, then R is a GCD domain.

\vfill\eject

\section*{Bibliography}

Abramov, Petkov\v sek,
On Polynomial Solutions of Linear Partial
Differential and (q-)Difference Equations.  CASC 2012.

\begin{quote}
Proves the impossibility of algorithmically solving PDEs in a polynomial
ring as a straightforward consequence of the undecidability
of Hilbert's tenth problem.
\end{quote}

Alin Bostan, Thomas Cluzeau, Bruno Salvy.
Fast Algorithms for Polynomial Solutions
of Linear Differential Equations. 2005.

\begin{quote}
{\tt http://specfun.inria.fr/bostan/publications/BoClSa05.pdf}

Explains the construction of the indicial polynomial and its
use in solving linear ODEs with polynomial coefficients.
\end{quote}

Saito, Sturmfels, Takayama, Groebner Deformations of Hypergeometric Equations.
Springer Verlag. 1999.

\begin{quote}
A standard reference work on D-modules.  Haven't read it because it's
too expensive.
\end{quote}

Harrison Tsai.  Algorithms for Algebraic Analysis.  Ph.D. thesis, UC Berkeley, 2000.

\begin{quote}
One of Sturmfels' students.  The D-module algorithms developed here
and in the previous reference require the module to be of finite
holonomic rank, which is generally not the case in the equations
of interest to us.
\end{quote}

\end{document}
