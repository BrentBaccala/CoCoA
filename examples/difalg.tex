
\documentclass{article}

\usepackage{vmargin}
\usepackage{times}
\usepackage{graphics}
\usepackage{amsmath, amscd, amsxtra, amsthm}
\usepackage{amssymb}
\usepackage{comment}
\usepackage{tikz}

\usepackage{supertabular}

%\usepackage[cspex,bbgreekl]{mathbbol}

%\setmarginsrb{0.5in}{0.5in}{0.5in}{0.5in}{0.5in}{0.2in}{0pt}{0pt}
\setmargnohfrb{0.5in}{0.75in}{0.5in}{0.5in}

% These settings are for eReader format.  Comment out for regular format.
%\setpapersize{A5}
%\setmargnohfrb{0in}{0in}{0in}{0.1in}

\newcommand{\Lp}[2]{\ensuremath{\text{L}^{#1}(#2)}}
\newcommand{\T}{\ensuremath{\mathbb{T}}}
\newcommand{\ip}[2]{\ensuremath{\langle #1, #2\rangle}}


%\newcommand{\C}{\ensuremath{\mathbb{C}}}
\newcommand{\C}{\ensuremath{{\bf C}}}
\newcommand{\R}{\ensuremath{{\bf R}}}
\newcommand{\GCD}{\ensuremath{\rm{GCD}}}

\pagestyle{empty}

% Get rid of section numbers
\def\thesection{}

% Get rid of page numbers
\def\thepage{}

\begin{document}

\parindent 0pt
\parskip 12pt

\section*{A Very Brief Introduction to Differential Algebra}

We'll consider
differential fields (resp. rings), which is are fields (resp. rings)
equipped with one or more additional unary operations (derivations),
each of which satisfy two axioms:

\begin{center}
   $D(a+b) = Da+Db$

   $D(ab) = (Da)b + a(Db)$
\end{center}

I'll often write a derivation as either a subscripted delta ($\delta_x$ instead of $D$),
a subscripted variable ($f_x$ instead of $Df$ or $\delta_x f$) or an apostrophe
(if only one derivation is being considered, like in a lemma).

{\bf Definition} An element $t$ of a differential field $K$ is {\it monomial} over
a subfield $k \subset K$ w.r.t a derivation $D$ if $t$ is transcendental
over $k$ and $Dt \in k[t]$.  Bronstein p. 91.

In simpler terms, the derivation of a monomial element is a
polynomial.  Most common transcendental field extensions are monomial.
For example, if $t = \ln x$, then $t_x = 1/x$, which is a polynomial
in $\C(x)[t]$, so $\ln x$ is monomial over rational functions in $x$ with
respect to derivation by $x$, as is $\exp x$, since if $t=\exp x$,
then $t_x=t\in \C(x)[t]$.  However, $t=\sqrt x$ is not monomial, even though $t_x
= t/2x \in \C(x)[t]$, since $\sqrt x$ is not transcendental (it's
algebraic), nor is $t= \exp(\sqrt x)$ monomial over $\C(x)$ w.r.t. $\delta_x$, since
$t_x = \sqrt x \exp(\sqrt x) / (2 x) \notin \C(x)[t]$.
However, $t=\exp(\sqrt x)$ is monomial over $\C(x,\sqrt x)$


{\bf Definition} Let $k$ be a differential field extended by a single
monomial $t$ to form a differential ring $k[t]$.
A polynomial $f \ in k[t]$ is {\it normal} with respect to a
derivation $D$ if $\gcd(f,Df)=1$ and {\it special} if $\gcd(f,Df)=f$.
Bronstein p. 92.

For an irreducible polynomial, these are the only two cases.
Reducible polynomials can be factored into a normal and a special
component.  If $Dt$ is constant (the ordinary case), all irreducible
polynomials in $k[t]$ are normal.

{\bf Lemma} An irreducible normal factor of a polynomial with multiplicity
$p$ and non-zero derivative appears in the polynomial's derivative
with multiplicity $p-1$.  {\bf Proof.}  Write the polynomial as $f^p q$, where
$\GCD(f,q)=1$.  Its derivative is:

$$p f^{p-1} f' q + f^p q' = f^{p-1}(pf'q + fq')$$

$f$ doesn't divide $f'$ (the factor is {\it normal}), and both $f'$
and $q$ are not zero, so $f$ doesn't divide either $pf'q$ or
$(pf'q+fq')$.

{\bf Lemma} An irreducible factor of a polynomial with multiplicity
$p$ and zero derivative appears in the polynomial's derivative with
multiplicity at least $p$, unless the derivative is zero.  Write the
polynomial as $f^p q$, where $\GCD(f,q)=1$.  Its derivative is $f^p
q'$, and while $f$ doesn't factor $q$, it might factor $q'$, or
$q'$ might be zero.

{\bf Lemma} An irreducible normal factor of a polynomial with multiplicity
$p$ appears in the polynomial's derivative with multiplicity at least
$p-1$, unless the derivative is zero.

\vfill\eject
\section*{A Prototype Problem}

%Let's consider a simplified, one-dimensional version of Schr\"odinger's equation:
Let's consider the one-dimensional heat equation:

$$\frac{\delta^2}{\delta x^2} \Psi = \frac{\delta}{\delta t} \Psi$$

We'll use two derivations, $\delta_x$ and $\delta_t$, so I'll often
write the previous equation as $\delta_x^2 \Psi = \delta_t \Psi$
or $\Psi_{xx} = \Psi_{t}$.

What field do we use?  Good question!

Let's start with the rational functions in $x$ and $t$ with
complex coefficients: ${\bf C}(x,t)$.  Are there any solutions
to $\delta_x^2\Psi = \delta_t\Psi$ in this differential field?

\begin{comment}

No.  Everything in ${\bf C}(x,t)$ can be written uniquely as a ratio
of polynomials in $x$ and $t$ with GCD 1.  Let's start by considering
a non-trivial denominator and looking at how it behaves under one of
our differential operators, say $\delta_x$:

$$\left(\frac{n}{d}\right)' = \frac{n'd - nd'}{d^2}$$

Treating $n$ and $d$ as polynomials in $x$ with coefficients in
$\C[t]$, we note that $\delta_x$ maps all of these coefficients to
zero, so $d'$ must have lower degree in $x$ than $d$, so $d$ can't
divide $d'$, though they might share a common factor (i.e, $x^2$
doesn't divide $2x$, but they share a common factor of $x$).  Since
$d$ shares no factors with $n$, $nd'$ shares no more than a subfactor
with $d^2$, and $nd'/d^2$, unless it is zero, must have its
denominator $x$-degree greater than $d$'s.  $n'd/d^2 = n'/d$ on the
other hand, can have a denominator $x$-degree no larger than $d$'s, so
applying $\delta_x$ to $n/d$ causes the $x$-degree of the denominator to
increase, unless the denominator didn't involve $x$ at all (the case
where $nd'$ would be zero).

%Now treating $n$ and $d$ as polynomials in $t$ with coefficients in
%$\C[x]$, we easily see that applying $\delta_x$ may change around the
%coefficients, but can't increase the $t$-degree of the denominator (it
%can only decrease).

I was hoping to now show that $\delta_x$ can't increase the $t$-degree
of the denominator, but that's not the case, i.e:

$$\delta_x\left(\frac{1}{xt-1}\right) = -\frac{t}{(xt-1)^2}$$

\end{comment}

\begin{comment}
No.  Let's start by considering how $\delta_x$, when applied to a
polynomial in ${\bf C}[x,t]$, affects its $x$-degree (the highest
power of $x$ in its monomials).

Treating $d$ as a polynomial in $x$ with coefficients in
$\C[t]$, we note that $\delta_x$ maps all of these coefficients to
zero, so $d'$ must have lower degree in $x$ than $d$, so $d$ can't
divide $d'$, though they might share a common factor (i.e, $x^2$
doesn't divide $2x$, but they share a common factor of $x$).
\end{comment}

Everything in ${\bf C}(x,t)$ can be written uniquely as a ratio of
polynomials in $x$ and $t$ with GCD 1, so let's consider an arbitrary
field element $\Psi = n/d$ with $n,d\in\C[x,t]$.  Applying our
differential operators (I now write $\delta_x n$ as $n_x$):

$$\delta_x \frac{n}{d} = \frac{n_x d - n d_x}{d^2}$$

$$\delta_x^2 \frac{n}{d} = \frac{n_{xx} d^2 - n d_{xx} d - 2 n_{x}d_{x}d + 2nd_x^2}{d^3}$$

$$\delta_t \frac{n}{d} = \frac{n_t d - n d_t}{d^2}$$

If $\Psi = \frac{n}{d}$ satisfies $\delta_x^2 \Psi = \delta_t \Psi$, then we must have:

$$n_{xx} d^2 - n d_{xx} d - 2 n_{x}d_{x}d + 2nd_x^2 = n_t d^2 - n d d_t$$

Rearranging:

$$2nd_x^2 - n d_{xx} d = n_t d^2 - n d d_t - n_{xx} d^2 + 2 n_{x}d_{x}d$$

\begin{comment}
Notice that $d$ divides everything on the RHS, so it must also divide the LHS.
We know $\GCD(n,d)=1$, by hypothesis, and we just concluded that
$d$ can't divide $d_x$, so it can't divide the LHS.  Contradiction.
\end{comment}

Consider $f$, a single irreducible factor of $d$,
appearing with multiplicity $p >= 2$, and $f_x \ne 0$, so $\delta_x$ drives
its order down by one.  $f$ appears in $d_x$ with power $p-1$ and in
$d_{xx}$ with power $p-2$, and doesn't appear in $n$ at all, since
$\GCD(d,n)=1$.  Thus, the terms on the LHS have $f$-order exactly
$2p-2$, while all of the terms on the RHS have $f$-order at least
$2p-1$ ($2p$, $2p-1$, $2p$, and $2p-1$, plus possible factors in $n$'s
derivatives, plus another factor in $d_t$ if $f$ does not
involve $t$):

$$2d_x^2 - d_{xx} d = m f^{2p-1}$$

Now let's expand $d=f^pq$:

$$d = f^p q \qquad d_x = f^{p}q_x +pf^{p -1}f_xq$$
$$d_x^2 = f^{2p}q_x^{2} +2pf^{2p -1}f_xqq_x +p^{2}f^{2p -2}f_x^{2}q^{2}$$
$$d_{xx} = f^{p}q_{xx} +2pf^{p -1}f_xq_x +pf^{p -1}f_{xx}q +(p^{2} -p)f^{p -2}f_x^{2}q$$

Substituting into $2d_x^2 - d_{xx} d = m f^{2p-1}$ and cancelling $f^{2p-2}$:

$$-f^{2}qq_{xx} +2f^{2}q_x^{2} +2pff_xqq_x -pff_{xx}q^{2} +(p^{2} +p)f_x^{2}q^{2} = m f$$

$$(p^{2} +p)f_x^{2}q^{2} = m f+f^{2}qq_{xx} -2f^{2}q_x^{2} -2pff_xqq_x +pff_{xx}q^{2}$$

$f$ factors the RHS, but $\GCD(f,q)=1$ and $\GCD(f,f_x)=1$ by
construction, so the LHS must be zero, $p^2 + p = 0$, implying that
$p$ is 0 or -1, both of which contradict the assumption that $p \ge
2$.  So $d$ can have no irreducible factors that involve $x$ and
appear with power greater than unity.

\begin{comment}
The leading term in $2nd_x^2 - n d_{xx} d$ is of power $x^{2r-2}$ and
has coefficient $2r^2 d_r - r(r-1)d_r^2$, so $r^2 + r =0$, and $r$
must be either $0$ or $-1$.  Obviously it can't be $-1$, what about
$0$?  Then $d$ would be a polynomial in $t$ that didn't involve $x$.
\end{comment}

We still have to consider the case of a square-free irreducible factor
involving $x$.  In this case, we can rearrange like this:

$$2nd_x^2 = n_t d^2 - n d d_t - n_{xx} d^2 + n d_{xx} d + 2 n_{x}d_{x}d$$

Note that $d$ divides the RHS, and a purely square-free factor of $d$
involving $x$ would not appear in $d_x$ or $n$, so the LHS must be
zero, which can only occur if either $n=0$ or if $d_x=0$, which
contradicts the assumption that $d$'s factor involves $x$.

To summarize, we've concluded that either $n=0$ or that $d_x=0$,
and thus be $d$ must be a polynomial in $\C[t]$.  In this case, the
derivatives $d_x$ and $d_{xx}$ vanish from the original equation and it becomes:

$$n_{xx} d^2 = n_t d^2 - n d d_t$$

Cancelling and rearranging:

$$n d_t = n_t d - n_{xx} d$$

$d$ can't divide its own derivative, and $\GCD(n,d)$ is 1 by
hypothesis, so this equation can only hold if $d_t$ is zero, so $d$
must be a constant (LHS), and $n_t = n_{xx}$ (RHS).  Analysing
this equation using the techniques later in this paper
\footnote{see {\it Solving Polynomial Differential Equations}}
shows that $n_t = n_{xx}$ admits an infinity of solutions
in $\C[x,t]$, with an infinite basis set $\{B_0, B_1, \ldots\}$:

$$B_{2i} = \sum_{j=0}^i \frac{(2i)!}{(2i-2j)!j!} x^{2i-2j} t^{j}
\qquad B_{2i+1} = \sum_{j=0}^i \frac{(2i+1)!}{(2i-2j+1)!j!} x^{2i-2j+1} t^{j}$$

% Maxima code
%
% Beven(i) := sum((2*i)!/((2*i-2*j)!*(j)!) *x^(2*i-2*j)*t^(j), j, 0, i);
%
% Bodd(i) := sum((2*i+1)!/((2*i-2*j+1)!*(j)!) *x^(2*i-2*j+1)*t^(j), j, 0, i);

$$\{1, x, x^2 + 2t, x^3 + 6xt, x^4 + 12 x^2t + 12 t^2, x^5+20x^3t+60xt^2, \ldots\}$$

So the only solutions to $\delta_x^2 \Psi = \delta_t \Psi$ in
$\C(x,t)$ are linear combinations of these polynomials.

Incidentally, what happens if we apply the irreducible factor analysis
to the numerator equation $n_{xx}=n_t$ with $n=f^p q$?  We find that
$p$ must be one, and an additional differential condition, namely,
that $f_{xx}q+2f_x q_x - f_t q$ be a multiple of $f$.  Notice that
$B_5 = x^5+20x^3t+60xt^2 =
x(x^2+(10+2\sqrt{10})t)(x^2+(10-2\sqrt{10})t)$,
\footnote{\tt expand(x*(x**2+(10 + 2 *sqrt(10))*t)*(x**2+(10 - 2*sqrt( 10))*t));}
and that any selection
of one of these factors as $f$ (and the product of the remaining two
as $q$) satisfies the differential condition.  While interesting as
the logical conclusion of the irreducible factor solution technique,
this information doesn't seem to lead directly to a solution.  Open
question: would an elimination analysis with differential Groebner
bases turn up additional relations on the irreducible factors?

Can we find a different field that contains more solutions to this PDE?

An extension field is the obvious choice.

It's a basic theorem in differential algebra that differentials extend
in a single unique way into an algebraic extension (Bronstein Theorem
3.2.3), while transcendental extensions require only that the
differential be specified for the primitive element that extends the
field (Bronstein Theorem 3.2.2).

Let's extend $\C(x,t)$ by $r$ (an algebraic) and $z$ (a
transcendental).  I'll set $r^2 = t$, so its differentials are:

$$\delta_x r = 0 \qquad \delta_t r = \frac{1}{2r} $$


Since $z$ is transcendental, I can pick (almost) anything for its differentials.  For
reasons that will become apparent later, let's try

$$\delta_x z = -\frac{x}{2t} z \qquad \delta_t z = \frac{x^2}{4t^2} z$$

Now, is there a solution to $\delta_x^2 \Psi = \delta_t \Psi$ in
$\C(x,t,r,z)$?  You bet!  Let's try $\Psi = z/r$.

$$\delta_x \Psi = - \frac{xz}{2rt}$$
$$\delta_x^2 \Psi = - \frac{2zt-x^2z}{4rt^2}$$
$$\delta_t \Psi =  \frac{x^2z - 2zt}{4rt^2} = \delta_x^2 \Psi$$

So, what's the point?  That I can concoct some screwball extension
that solves a PDE?  Actually, $z$ is just an exponential that solves
$\frac{\delta z}{\delta y}=z$ for $y=-x^2/(4t)$, and those weird
differentials are simply the result of applying the chain rule
to compute $\frac{\delta z}{\delta x}$ and $\frac{\delta z}{\delta t}$.
Analytically, we would write this solution as:

$$\Psi(x,t) = \frac{e^{-\frac{x^2}{4t}}}{\sqrt{t}}$$

Later, I'll use differential algebra to show that this is the only
solution (times an arbitrary constant) in $\C(x,t,r,z) \backslash \C(x,t)$.

The two main extension types I'm interested in are algebraic
extensions (since they've been so heavily studied), and ``holonomic''
extensions of the form $\frac{\delta z}{\delta r} = f$ for $r$ and
$f$ arbitrary field elements.  This models first order semilinear ODEs.

Then we can start asking questions like, given a finite number of
algebraic and holonomic extensions, can we find solutions to a given
PDE?  This corresponds to asking whether we can solve a PDE by
breaking it down into algebraic functions and ODEs, so would subsume
separation of variables as a special case.

How can we deal with boundary conditions?  Well, once we've found all
solutions to a PDE in a particular extension field, we can then
restrict them to the boundary (say, $t=1$, in the example above), and
ask whether they form a basis for whatever function space (typically
$\text{L}^2$) our boundary condition exists in.  Since exponentials
(as we saw above) are a simple case of the holonomic extension,
Fourier analysis would be subsumed as a special case.

It would be nice to have theorems telling us what conditions are
needed to get a basis set in some extension field, and of course how
to setup that extension and compute the basis elements.

Is there any hope of proving such theorems?  Well, since we're working
with algebra, we've got the machinery of algebraic geometry available
(not something you typically expect in PDE theory).  Reducing modulo
$p$, for example, is definitely in the mix, though this suggestion
does has the flavour of an Army captain musing about tactical nukes
while contemplating an enemy bunker.

If successful, this program should result in a solution technique for
PDEs that would generalize both Fourier analysis and separation of
variables, so I think this is quite promising!


\vfill\eject
\section*{Solving $\Psi_{xx}=\Psi_t$ in $\C(x,t,z)$}

What about the intermediate field $\C(x,t,z)$?  Remember

$$z_x = -\frac{x}{2t} z \qquad z_t = \frac{x^2}{4t^2} z$$

{\bf Lemma}  For all $f \in \C[x,t,z], f \notin \C[x,t] \implies f_x \ne 0$.
Consider $f = \sum f_i z^i$.  Then:

$$f_x = \sum \left( f_{ix} - \frac{ix}{2t} f_i \right) z^i$$

If $f_x = 0$, then each of these terms must be 0.  The $x$-degree of $f_{ix}$
is one less than the $x$-degree of $f_i$, but the $x$-degree of $xf_i$ is
one greater than that, so either $i$ or $f_i$ is zero.  So all of the $f_i$
are zero except $f_0$, but then $f$ would be in $\C[x,t]$.  QED.

Returning to the original problems, we're once again led to consider the equation:

$$n_{xx} d^2 - n d_{xx} d - 2 n_{x}d_{x}d + 2nd_x^2 = n_t d^2 - n d d_t$$

This time, we'll split $d$ into its normal and special components,
with respect to $\delta_x$:

$$d =d_s d_n$$

This time, we consider an irreducible normal factor $f$ of $d_n$ with
non-zero $x$-derivative, and use the same logic as before to conclude
that $f_x$ is zero, so $d_{nx}$ is zero.  By the previous lemma,
$d_{n}$ must be in $\C[t]$.

Next we attack $d_s$, using Bronstein's Theorem 5.1.2 (p. 130) which
says that $d_s$ has to be a power of $z$.

$$d=z^a f \qquad f \in \C[t]$$

$$d_x = a z^a f \left( - \frac{x}{2t} \right)$$

$$d_{xx} = a^2 z^a f \left( - \frac{x}{2t} \right)^2 - a z^a f \left( \frac{1}{2t} \right)$$

$$d_t = z^a \left( a \frac{x^2}{4t^2} f + f_t\right)$$

Our polynomial equation (after cancelling $f$ and $z^{2a}$, and clearing the denominator) becomes:

$$(a^{2} +a)x^{2}fn +4axftn_x +4ft^{2}n_{xx} -4ft^{2}n_t +2aftn +4f_tt^{2}n = 0$$

This implies that $4f_tt^{2}n$ must be a multiple of $f$, and since $f$ is irreducible and normal,
and $\gcd(f,n)=1$, this can only occur if $f=t$.

So now let's consider a denominator of the form $z^p t^a f$, where $f$ has no $t$ factor:

$$(p^{2} +p)x^{2}fn +4pxtfn_x +4t^{2}fn_{xx} -4t^{2}fn_t +4t^{2}f_tn +(2p +4a)tfn = 0$$

Again the minimum power of $f$ appears multiplied by $4f_tt^{2}n$, which is a contradiction
(since there's no $t$ factor in $f$), implying that $f_t$ is zero.

So we've reduced the denominator to the form $z^a$.  Expanding the polynomial equation we get:

$$(a^{2} +a)x^{2}n +4axtn_x +4t^{2}n_{xx} -4t^{2}n_t +2atn = 0$$

Now we look at the powers of $t$ and conclude that $(a^{2}
+a)x^{2}n$ must be a multiple of $t$, implying that the numerator
must include a $t$ factor.  Setting the numerator to be $t^b n$, and
leaving the denominator as $z^a$, we obtain:

$$(a^{2} +a)x^{2}t^{b}n +4axt^{b +1}n_x +4t^{b +2}n_{xx} -4t^{b +2}n_t +(2a -4b)t^{b +1}n = 0$$

So, $(a^{2} +a)x^{2}n$ must be zero or a multiple of $t$, which implies that $a^2+a$ must be
zero, so $a$ must be 0 or -1.  Since it can't be -1, it must be zero,
and the denominator is therefore trivial.

Now let's look at the numerator, assume that it has the form $\sum n_i z^i$, where the $n_i$
are polynomials in $\C[x,t]$.  Expanding out $n_{xx} - n_t = 0$, we obtain:

$$\sum_i \left[ (i^{2} -i)x^{2}n_i -4ixtn_{ix} +4t^{2}n_{ixx} -4t^{2}n_{it} -2itn_i \right] z^i = 0$$

Looking at the powers of $t$, we conclude that $(i^{2} -i)x^{2}n_i$ must be a multiple
of $t$, so we now try a numerator of the form $\sum n_i t^a z^i$, and again
conclude that $(i^2-i) x^2 n_i$ must be a multiple of $t$, which implies that $i^2-i=0$
and thus $i$ is either zero or one.

Our numerator equation for $i=1$ assumes the form:

$$-2n_x x - n + 2 t n_{xx} = 2 t n_t$$

Assuming that $n$ has the form $\sum a_b t^b$, where the $a_b$ are polynomials in
$\C[x]$, we expand this into:

$$-2\sum x n_{bx} t^b - \sum a_b t^b + 2 \sum a_{b-1xx} t^{b} = 2 \sum b a_b t^b$$

The $b=0$ term becomes $-2xa_{0x} - a_0 = 0$.  Since $a_0$ is a polynomial
in $\C[x]$, this equation implies that each of its coefficients would have
to be at least twice itself, which is impossible, so $a_0=0$.  Likewise,
$a_0=0$ implies that $a_1$'s equation is $-2xa_{1x}-a_1=2a_1$, which is
likewise impossible.  By induction, $a_b=0$ implies that $a_{b+1}=0$,
so all of the $a_b$ are zero and there is no $z$ ($i=1$) term in
the numerator.

The numerator thus reduces to the $i=0$ term, for which $n_{xx}=n_t$
and we have the same solution set as before.

\vfill\eject
\section*{Solving $\Psi_{xx}=\Psi_t$ in $\C(x,t,r,z)$}

We should be able to find more solutions in $\C(x,t,r,z)$, where $r^2=t$.

Let's study this field as the fraction field of the ring
$\C(x,t,r)[z]$.  This enables us to cleanly analyse $z$ as a monomial
extension.  Any polynomial in $z$ will have a non-zero $x$-derivative
(previous lemma), and we've already dealt with the denominator cases for
normal factors with $f_x \ne 0$, so we once again consider a special
denominator of the form $z^a$, and a numerator of the form $t^b n +
t^c n_r r$ where both $n$ and $n_r$ are in $\C(x,t)$.  This leads
to a pair of equations:

$$(a^{2} +a)x^{2}t^{b}n +4axt^{b +1}n_{x} +4t^{b +2}n_{xx} -4t^{b +2}n_{t} +(2a -4b)t^{b +1}n = 0$$

$$(a^{2} +a)x^{2}t^{c}n_r +4axt^{c +1}n_{rx} +4t^{c +2}n_{rxx} -4t^{c +2}n_{rt} +(2a -4c -2)t^{c +1}n_r = 0$$

Since $t$ can not factor either $n$ or $n_r$, $a^2+a$ must be zero, again
leading to the conclusion that $z$ does not appear in the denominator.

The numerator analysis, based on $\sum n_i t^a z^i$ is unchanged, leading
again to the conclusion that $i$ is either zero or one.  The two resulting
equations are:

\begin{align*}
n_{xx} = n_t   & \qquad (i=0) \\
-2n_x x - n + 2 t n_{xx} = 2 t n_t  &  \qquad (i=1)
\end{align*}

This time our solution space is $\C(x,t,r)$, however.  Assuming $n$
has the form $n + n_r r$ (remember that no fraction field is needed
for algebraic extensions), we obtain four equations to be solved in $\C(x,t)$:

\begin{align*}
n_{xx} = n_t  & \qquad (i=0) \\
n_r = 2tn_{rxx} -2tn_{rt} & \qquad (i=0; r) \\
-2xn_{x} +2tn_{xx} -2tn_{t} -n = 0 & \qquad (i=1) \\
-2xn_{rx} +2tn_{rxx} -2tn_{rt} -2n_r = 0 & \qquad (i=1; r)
\end{align*}

These four rational equations convert to four polynomial equations:

$$-ndd_{xx} +ndd_t +2nd_x^{2} -2n_xdd_x +n_{xx}d^{2} -n_td^{2} = 0$$
$$-2tndd_{xx} +2tndd_t +4tnd_x^{2} -4tn_xdd_x +2tn_{xx}d^{2} -2tn_td^{2} -nd^{2} = 0$$
$$2xndd_x -2xn_xd^{2} -2tndd_{xx} +2tndd_t +4tnd_x^{2} -4tn_xdd_x +2tn_{xx}d^{2} -2tn_td^{2} -nd^{2} = 0$$
$$2xndd_x -2xn_xd^{2} -2tndd_{xx} +2tndd_t +4tnd_x^{2} -4tn_xdd_x +2tn_{xx}d^{2} -2tn_td^{2} -2nd^{2} = 0$$

The first one we've analysed already; its solutions are all in $\C(x,t)$.

Analyzing the remaining three using the techniques already discussed
(irreducible factors) reveal that their denominator's $x$-derivatives
are zero.  Setting $d_x$ and $d_{xx}$ to zero, we obtain:

$$2tnd_t +2tn_{xx}d -2tn_td -nd = 0$$
$$-2xn_xd +2tnd_t +2tn_{xx}d -2tn_td -nd = 0$$
$$-2xn_xd +2tnd_t +2tn_{xx}d -2tn_td -2nd = 0$$

Considering the next one, we see that $2tnd_t$ must be a multiple of $d$, so we're led to
consider $t$ as a factor of $d$, i.e, $d=t^a q$:

$$2nt q_t + 2tn_{xx}q -2tn_tq +(2a-1)nq = 0$$

which requires either $2ntq_t$ to be a multiple of $q$ (impossible), or $a=\frac{1}{2}$,
so the second equation has no solution.

The next-to-last equation also requires either $2ntq_t$ to be a multiple of $q$ or $a=\frac{1}{2}$, so it's also impossible.

The last equation becomes:

% $$-2xn_xt^aq +2at^anq + 2 t^{a+1} n q_t +2n_{xx}t^{a+1}q -2n_tt^{a+1}q -2nt^aq = 0$$

$$-2xn_xq  + 2 n t q_t +2tn_{xx}q -2tn_tq +(2a-2)nq = 0$$

At first, this appears to require $2ntq_t$ to be a multiple of $q$, but there is another possibility!
If $n$ and $q$ are both constant, then the equation reduces to:

$$(2a-2)nq = 0$$

which can be satisfied if $a=1$, which means $d=tq$.

In short, we've found another solution: $zr/t$, as expected from analysis.
This is the only solution in $\C(x,t,r,z) \backslash \C(x,t)$.

\vfill\eject
\section*{Solving $\Psi_{xx}=\Psi_t$ with a free exponential}

In previous sections, I took $z=\exp(-\frac{x^2}{4t})$, but this
requires too cleaver a guess to be useful as a general technique.
What happens if we relax this relationship, assume only that $z$ is an
exponential of some element in $\C(x,t)$, and look for solutions in
$\C(x,t,z)$?

We begin our analysis as before by concluding that there can be no
normal irreducible factors in the solution's denominator, so next
let's consider a solution of the form $\frac{n}{z^p}$, with an
exponential of the form $z=\exp(\frac{n_e}{f^a d_e})$.  Expanding out
our equation, we find that the only term not involving $f$ is
$p^{2}a^{2}f_x^{2}n_e^{2}d_e^{2}n$.  $f$ is coprime to $n_e$ and $d_e$,
and let's add the assumption that it's also coprime to $n$.  If $p$
and $a$ are both non-zero, we must have $f_x$ zero, so adding that
assumption produces an equation of the form:

$$[\cdots]f^{2a+1} + [\cdots]f^{a+1} + [\cdots]f^a + [\cdots]f = 0$$

Here, for the first time, we find exponents that are {\it incomparable},
since we can't separate the final two terms without additional
assumptions on the value of $a$.  Assuming that $a$ is at least two,
the $f$ terms are:

$$p^{2}n_e^{2}d_{ex}^{2}n -2p^{2}n_en_{ex}d_ed_{ex}n +p^{2}n_{ex}^{2}d_e^{2}n = p^2(n_e d_{ex} - n_{ex} d_e)^2n = p^2(n_e d_e)_x^2n$$

which must be a multiple of $f$ (or zero), so we integrate and
conclude that $n_e d_e = f^b q+c$, where $c_x=0$.  This makes
$p^2(n_e d_e)_x^2n$ a multiple of $f^{2b-2}$.  If we now assume
that $2b-1 > a$, then the $f^a$ term dominates and $paf_tn_ed_e^{3}n$
must be a multiple of $f$, which is impossible.  So $a>=2$ and $2b+1>a$
(along with the coprimality assumptions) is an impossible combination,
and we can begin to build a {\it case tree} of possibilities.

What about $a>=2$ and $2b-1=a$?  Then we've insured that the
$f$ and $f^a$ terms have equal powers of $f$, so we can
move on and conclude that the two terms together must also
be a multiple of $f$ (to cancel the $f^{a+1}$ term):

$$p(n_e d_e)_x^2 - af_tn_ed_e^{3} = rf$$

We might think that $r$ and $f$ must be coprime, since $f^a$ must be
multiplied by exactly $f$ to match $f^{a+1}$, but this neglects the
possibility that the $f^{a+1}$ coefficient might itself be a multiple
of $f$, in fact has to be to match $f^{2a+1}$.

Finally, we consider the $f^{a+1}$ term and conclude that it also has
to be a multiple of $f$, in fact exactly a multiple of $f^a$ in order
to cancel the $f^{2a+1}$:

$$2n_ed_ed_{ex}n_x +n_ed_ed_{exx}n -n_ed_ed_{et}n -2n_ed_{ex}^{2}n -2n_{ex}d_e^{2}n_x +2n_{ex}d_ed_{ex}n -n_{exx}d_e^{2}n +n_{et}d_e^{2}n = s f^a$$



\vfill\eject
\section*{Solving Polynomial Differential Equations}

Often we're confronted with an equation like $n_{xx} = n_t$, where $n$
is constrained to be a polynomial.  A fair amount is known about
such equations, but there are still big gaps.

An important result is the Abramov-Petkov\v sek theorem, which proves
the impossibility of algorithmically solving PDEs in a polynomial ring
as a straightforward consequence of the undecidability of Hilbert's
tenth problem.  The proof is based on the close correlation between
solving polynomial differential equations and solving integer Diophantine
equations.  Abramov and Petkov\v sek demonstrated a simple
construction that allows any polynomial equation to be converted
into a PDE with the property than any polynomial solution to the PDE
can be trivially lifted to a solution of the original polynomial.
Thus, any algorithm that could solve arbitrary PDEs in a polynomial
ring could be used to solve arbitrary Diophantine equations, which
is impossible (the MRDP theorem).

However, their proof technique suggests to me that the degree of the
resulting Diophantine equations should be bounded by the order of the
PDE.  PROOF NEEDED.  If true, then this is good news since Sch\"odinger's
equation is second order and Cohen's GTM 239 suggests that second
degree Diophantine equations may be completely solvable.  See
section 6.3 (p. 341) in GTM 239.

The form of the Diophantine equations can be further restricted.
Since there are no cross derivatives of the form $\delta^2/\delta x \delta y$,
there should be no mixed monomials in the Diophantine equations,
so they can be separated into a linear component (solvable using
matrix reduction into Hermite normal form) and a sum of squares.
If the squares all have the same sign, which is likely since
the second derivatives in Schr\" odinger's equation are invariant
under exchange of variables, then the equation is of elliptic
type, meaning that it describes an n-dimensional ellipsoid
in real space $\R^n$, so its integer solutions are bounded
and thus computable.  Different values from the linear
component, however, might lead to an infinite number of
elliptic quadratics with unbounded total size, and it
also seems likely that a single PDE could generate an
infinite number of Diophantine equations.

Systems of partial differential equations have been extensively
studied in the form of D-modules, and algorithms have been developed
to find their polynomial and rational solutions in the case of finite
holonomic rank.  Most of our equations are of infinite rank, however.
The known D-module algorithms for rational solutions depend on finding
an algebraic variety (the singular locus), which is basically the
Zariski closure of the solution's singularities and thus provides
crucial information about what factors can be present in the
denominator.  This seems difficult to generalize into transcendental
extension fields, since the relationship between $x$ and $e^x$, for
example, is not algebraic.  Futuremore, Tsai's Lemma 2.1.5 states that in
the infinite rank case, the singular locus is trivial (it's the entire space).

How can we analyse the infinite rank case?  Let's consider a simple
case: the operator $\delta_x^2 - \delta_t$ acting on $\C[x,t]$.
Any polynomial in this ring can be expressed in the form $\sum c_{m,n} x^m t^n$,
so applying the operator we obtain:

$$\sum c_{m,n} m(m-1) x^{m-2} t^n - \sum c_{m,n} n x^m t^{n-1} = 0$$

Changing variables, we can collapse the sums together:

$$\sum c_{m+2,n} (m+2)(m+1) x^m t^n - \sum c_{m,n+1} (n+1) x^m t^n = 0$$
$$\sum \left[  (m+2)(m+1) c_{m+2,n} -  (n+1) c_{m,n+1} \right] x^m t^n = 0$$

This equation specifies relationships between the various $c_{m,n}$
coefficients.  For example, the $xt$ term specifies a relationship
between $c_{3,1}$ and $c_{1,2}$, specifically, $3c_{3,1}=c_{1,2}$.
The $x^3$ term specifies a relationship between $c_{5,0}$ and $c_{3,1}$,
$20c_{5,0} = c_{3,1}$.
If we chart all of the $c_{m,n}$ coefficients, we can graph these
relationships as follows:

\def\tikzcoefficientgraph{
\pgftransformscale{0.5}
%\draw (0,0) -- (5.5,0);
%\draw (0,0) -- (0,5.5);
\foreach \x in {0,...,5}
   \foreach \y in {0,...,5}
      \draw (\x,\y) node {.};

\foreach \x in {2,...,5}
   \draw (\x,-1) node [anchor=base] {$x^\x$};
\draw (1,-1) node [anchor=base] {$x$};
\draw (0,-1) node [anchor=base] {1};
\foreach \y in {2,...,5}
   \draw (-0.5,\y) node {$t^\y$};
\draw (-0.5,1) node {$t$};
\draw (-0.5,0) node {1};
}

\begin{center}
\begin{tikzpicture}

\tikzcoefficientgraph

\draw (1,2) node {x};
\draw (3,1) node {x};
\draw (5,0) node {x};
\draw (5,0) -- (3,1) -- (1,2);

\end{tikzpicture}
\end{center}

A dimension can be associated with the differential operator,
corresponding to the dimensions spanned in the graph.  The example
shows a one-dimensional operator, spanning a line.  Such an operator
generates an infinite family of {\it potential} solutions, each
corresponding to a line of the appropriate slope in the chart.  Two
subcases arise.  For a line of negative slope, as in the illustrated
example, boundary conditions exist on both ends of the line, as the
coefficients must become zero as the line passes out of the first
quadrant.  These conditions must be satisfied for the potential
solution to be an actual solution.  In the other case, non-negative
slope, only one boundary condition exists on each line, and a
recursion relationship gives rise to an infinite sequence of
coefficients on each line.  An example is the operator $\delta_x^2+t$,
which graphs like this:

\begin{center}
\begin{tikzpicture}

\tikzcoefficientgraph

\draw (1,2) node {x};
\draw (3,3) node {x};
\draw (5,4) node {x};
\draw (5,4) -- (3,3) -- (1,2);

\end{tikzpicture}
\end{center}


For a two-dimensional operator, such as $\delta_x^2+\delta_x-\delta_t$,
all of the coefficients are linked together and can not be separated.
For a zero-dimensional operator, such as $x\delta_x + t\delta_t$,
each coefficient is independent of all others, and a Diophantine
equation determines if each coefficient is an admissible solution.
Abramov and Petkov\v sek used zero-dimensional operators to tie
this problem to Hilbert's tenth problem.
These two examples are illustrated as follows:

\begin{center}
\begin{tikzpicture}

\tikzcoefficientgraph

\draw (1,2) node {x};
\draw (2,1) node {x};
\draw (3,1) node {x};

\draw (1,2) -- (2,1) -- (3,1) -- (1,2);

\pgftransformscale{2}
\pgftransformxshift{200}

\tikzcoefficientgraph

\draw (3,3) node {x};

\end{tikzpicture}
\end{center}

Polynomial rings with more indeterminates give rise to more possibilities.
Operators can have dimension from zero up to the number of indeterminates.

The operator $x^a \delta_x^b$ acts on $c_{m,n} x^m t^n$ to produce
$m_{(b)} c_{m,n} x^{m-b+a} t^n$, which transforms to $(m+b-a)_{(b)}
c_{m+b-a,n} x^m t^n$, where $m_{(b)}$ is the Pochhammer falling
factorial.  Applying this operation on each term in a differential
operator transforms the operator into a recursion relationship on the
coefficients.

For example, the operator $\delta_x^2 - \delta_t$ transforms as:

$$\delta_x^2 - \delta_t \Longrightarrow (m+2)(m+1)c_{m+2,n} - (n+1)c_{m,n+1}$$

Treating $m$ as the horizontal coordinate and $n$ as the vertical, we
see from this expression that the operator is one-dimensional and has
slope $-\frac{1}{2}$, since it contains the points $(m+2,n)$ and
$(m,n+1)$.

Having identified the dimension of the operator, it is now profitable
to change coordinates into a system that separates dimensions within
the operator's hyperplane and dimensions orthogonal to it.  Thus, for
an $m$-dimensional operator in an $n$-dimensional polynomial ring, we
seek to find $n-m$ dimensions to identify $m$-dimensional hyperplanes,
and $m$ dimensions within the hyperplanes.

In our example, we can characterize the lines by introducing
$p$ as the $m$-intercept, and thus the highest $x$ power appearing in
the line's solution polynomial.  We also introduce $a$ as a coordinate
along the line, with $a=0$ corresponding to the point on the $m$-axis,
and higher values of $a$ moving upwards and left along the line.

Setting $m=p-2a$ and $n=a-1$, we can transform the coefficients and
their recursion relationship as follows, calling them $d$'s in the new
numbering system:

%$$c_{m,n+1} = c_{p-2a,a} = d_{p,a}$$
%$$c_{m+2,n} = c_{p-2(a-1),a-1} = d_{p,a-1}$$
%$$(p-2a+2)(p-2a+1) d_{p,a-1} = a d_{p,a}$$
$$c_{m,n+1} = c_{p-2a,a} = d_{p,a}  \qquad c_{m+2,n} = c_{p-2(a-1),a-1} = d_{p,a-1}$$
$$(m+2)(m+1)c_{m+2,n} - (n+1)c_{m,n+1} = 0 \Longrightarrow (p-2a+2)(p-2a+1) d_{p,a-1} = a d_{p,a}$$

The new numbering clearly relates coefficients along a single line, as
evidenced by the $p$ subscript remaining unchanged through the
recursion relationship.

Now, in order to have a polynomial and not an infinite sum, we must
have clear starting and ending points, beyond which the coefficients
are all zero.  A starting point requires $d_{p,a}$ to be non-zero,
while $d_{p,a-1}$ is zero, so $a$ must be zero.  Likewise, an ending
point requires $d_{p,a}$ to be zero, while $d_{p,a-1}$ is non-zero,
so $(p-2a+2)(p-2a+1)$ must be zero.  Since $p$ and $a$ are both
integers, this requires (in general) solving Diophantine equations,
which comes as no surprise by now!

These specific Diophantine equations are easy, though.  $a=0$ is the
only possible starting point, and the only possible end points are
$a=\frac{p}{2}+1$, if $p$ is even, or $a=\frac{p+1}{2}$, if $p$ is
odd.  We also check to ensure that the resulting $m,n$ coordinates
fall in the first quadrant (they do), to prevent any negative powers
from appearing in our ``polynomials''.

Therefore, for each value of $p$, we can set the value of $d_{p,0}$
arbitrarily, and with further values of $d_{p,a}$ being determined
by the recursion relationship:

$$d_{p,a} = \frac{(p-2a+2)(p-2a+1)}{a} d_{p,a-1}$$

Combining the recursions, we obtain:

$$d_{p,a} = \frac{p(p-1)\cdots(p-2a+2)(p-2a+1)}{a!} d_{p,0} = \frac{p!}{(p-2a)!a!} d_{p,0}$$

Earlier in this paper, I separated the even and odd values of $p$ and presented the solutions as:

$$B_{2i} = \sum_{j=0}^i \frac{(2i)!}{(2i-2j)!j!} x^{2i-2j} t^{j}
\qquad B_{2i+1} = \sum_{j=0}^i \frac{(2i+1)!}{(2i-2j+1)!j!} x^{2i-2j+1} t^{j}$$

Let's consider another example, the operator $-2x\delta_x - 1 +2t\delta_x^2 - 2t\delta_t$,
which transforms to the recursion:

$$-2m c_{m,n} - c_{m,n} + 2m(m-1)c_{m-2,n+1} - 2nc_{m,n}$$

In this form, we see clearly that this is a one-dimensional operator, even though it
has four terms.  Setting $m=p-2a$ and $n=a$, we transform to:

$$(-2p+4a-1-2a)d_{p,a} + 2(p-2a)(p-2a-1)d_{p,a+1}$$

The problem lies in the first factor, $-2p+4a-1-2a$, which can never be zero for
any integer values of $p$ and $a$, so our ending condition is never satisified.
Thus, this operator has no polynomial solution in $\C[x,t]$.

For two-term recursions like these two examples, zeroing the
Diophantine equations provides both necessary and sufficient
conditions for the recursion to terminate properly.  For higher order
one-dimensional recursions, this technique provides necessary
conditions that might not be sufficient.  Consider a three-term
recursion involving $d_a$, $d_{a+1}$, and $d_{a+2}$.  For the sequence
to start at $d_b$, $d_{a+2}$'s coefficient must be zero when $a=b-2$,
since both $d_{b-1}$ and $d_{b-2}$ will be zero.  Likewise, for the
sequence to end at $d_c$, $d_a$'s coefficient must be zero when $a=c$,
since $d_{c+1}$ and $d_{c+2}$ both must be zero.  This does not,
however, ensure that the recursion generates zero values for $d_{c+1}$
and $d_{c+2}$.

Higher dimension operators are more complex, but the same basic ideas
apply.  Consider the convex hull of the operator.  Zeroing the
coefficient of each term on the convex hull produces a Diophantine
equation that must be satisfied for that point to be in the solution
set, while all the other points are outside it.  The zeros of the
Diophantine equations define boundary hyperplanes that the polynomial
solution must lie within.

Theorem.  (Hopefully) Given any finite set of points on an
$n$-dimensional integer lattice, an $n$-dimensional convex polytope
whose vertices also lie on the integer lattice, and a distinguished
vertex on the convex polytope, an integer translate of the convex
polytope can be found such that the distinguished vertex lies
over the only point in the point set that is within the translated
convex polytope.  PROOF NEEDED.

Furthermore, after changing coordinates as above, we can develop a
system of Diophantine equations that may be more restrictive than the
individual equations.  For example, consider a 2-dimensional operator
with three vertices on its convex hull in a 5-dimensional space.  Each
vertex generates a Diophantine equation of the form $A(m,n,p,a,b)$
where $m,n,p$ identify a two-dimensional plane in the 5-dimensional
space, and $a,b$ identify a point on the plane.  Since any single
solution must be confined to a single plane, we are led to a
system of Diophantine equations of the form:

$$A(m,n,p,a,b) = 0$$
$$B(m,n,p,a',b') = 0$$
$$C(m,n,p,a'',b'') = 0$$

A simultaneous solution to this system identifies a single plane,
along with three boundary lines on that plane that confine a
polynomial solution.  Any single solution confines a finite number of
points, which can then be tested to see if they can actually be
assigned values that satisfy the recursion, requiring nothing more
complicated than solving a system of linear equations with constant
coefficients.  There may, however, be an infinity of solutions to the
system of Diophantine equations.



\vfill\eject
\section*{Operating in K[Z[p]]}

Often we need to operate in a monoid ring like $K[Z[p]]$, i.e, a
polynomial ring whose exponents are polynomials.  For example:

$$x^{2p}+2x^p+1=(x^p+1)^2$$

To order the monomials, an ordering is required on the exponents.  I use a ``high''
ordering where the coefficient of the highest power determines the comparison to zero.
\footnote{\tt http://math.arizona.edu/$\sim$rwilliams/math415A-fall2013/Ordered\_Rings\_and\_Fields.pdf}
Only monoid elements greater than zero are allowed as exponents, so $x^{p-50}$ is
in $K[Z[p]]$, but $x^{-1}$ is not.

We can't use Buchberger's algorithm directly, because it may not terminate since
the ring is not Noetherian.

$$(x^p) \subset (x^{p-1}) \subset (x^{p-2}) \subset \cdots$$

is an ascending chain of ideals that never stabilises.
For the same reason, $K[Z[p]]$ is not a unique factorization domain:

$$x^p = x^{p-10}x^{10} = x^{p-20}x^{20} = \cdots$$

However, $K[Z[p]]$ is a GCD domain.  PROOF NEEDED.

We can compute GCDs (needed for normalisation in the fraction field) using a
simple modification to Buchberger's algorithm.  Introduce new indeterminates
for each combination of lower and upper indeterminates.

$$x^p  \to  a$$
$$x^q  \to  b$$
$$x^{(p+q)} \to ab$$

This produces new polynomials with only integer exponents.  We can now
compute a Gr\"obner basis, which is how CoCoA computes GCDs, then map
the results back into the original ring.

How do we deal with monomials like $x^{p-1}$?  We can't factor it like
$x^px^{-1}$ because $x^{-1}$ isn't in our ring.  Instead, we map
$x^{p-1} \to a$ and then write $x^{p-1}$ as $a$ and $x^p$ as $xa$.
This implies that we can't construct a single derived ring that can
handle all polynomials in $K[Z[p]]$.  Instead, we construct a
derived ring for any particular GCD calculation.

The derived ring is Noetherian, but is not isomorphic to $K[Z[p]]$.
Instead, it is isomorphic to a subring of $K[Z[p]]$ and we can adjust
the construction so that the subring encompasses any finite number of
elements from $K[Z[p]]$.

Theorem.  (Hopefully) Given a ring R, if every pair of elements from R can be
embedded in a subring that is a GCD domain, then R is a GCD domain.
PROOF NEEDED.

\vfill\eject
\section*{Representing irreducibility with Gr\"obner bases}

\centerline{\tt http://mathoverflow.net/questions/217402}

We have a ring $R=K[x_1,\ldots,x_n]$ and a ring $S=K[y]$.  We want to
treat the $x_i$s as polynomials in $y$, so we're looking for a mapping
$f:R\to S$ that sends each $x_i$ to a polynomial in $y$ and satisfies
a system of polynomial equations $P$ in the $x_i$.  In other words, we
want $f$ to be a ring homomorphism that sends $I(P)$ to 0.  $f$, the
map from $R$ to $S$, is the solution we seek.

Now we want to impose an additional condition: a subset
$x_1,\ldots,x_i$ must map to irreducible elements $y_1,\ldots,y_i$ in
$S$.  Since $y_1,\ldots,y_i$ are irreducible, they are prime (in $S$),
so we quotient with respect to their ideal $I$ and get a quotient ring
$S/I$ that is an {\it integral domain}.  We can also quotient $R$ by the
ideal generated by $x_1,\ldots,x_i$ (call it $J$), and get $R/J$.  $f$
can be similarly restricted, and now we have a homomorphism $\hat{f}:
R/J \to S/I$.  We can construct a Gr\"obner basis for $R/J$ by appending
the $x_i$ that must be irreducible to the original system $P$, and
reducing $P \cup \{x_1,\ldots,x_i\}$ to a Gr\"obner basis.  This new
Gr\"obner basis gives relationships satisfied by the *equivalence
classes* in $S/I$.  ``Equal to zero'' in this quotient system means
``equal to zero or a multiple of an irreducible element'' in the
original system. However, if the quotient system is inconsistent, then
the original system is also inconsistent, at least subject to the
restriction that $x_1,\ldots,x_i$ must map to irreducibles.

Can we find additional relationships?  Surprisingly, yes!  We run this
calculation with each irreducible individually.  Pick one
$x_1,\ldots,x_i$, call it $x_j$, compute a quotient Gr\"obner basis for
$P \cup \{x_j\}$, take each polynomial in the quotient system's
Gr\"obner basis and test to see if it's in the original system.  If so,
then it's really equal to zero.  Otherwise, it's a multiple of $x_j$
and we can add that polynomial to the original system, equating it a
term of the form $m x_j$, with $m$ a new indeterminate.

The augmented system will have extraneous zeros, at least if we
require the irreducible polynomials to be non-zero.  We can handle
this by computing a primary decomposition and throwing away any
primary components that include an irreducible element among their
zeros.  This is the ideal-theoretic equivalent of factoring a
polynomial that must be equal to zero and throwing away factors that
we know are non-zero.  We can keep repeating these two processes
(quotient ring basis and primary decomposition) until our ideal
stabilizes.

{\bf Example}

Consider the equation $af^2+bf+c=0$, with $f$ restricted to be irreducible.

Step 1: Form the system $\{af^2+bf+c, f\}$ and reduce to the Gr\"obner
basis $\{f,c\}$.  Of course $f$ is here; our interest is $c$.  Since
it isn't in the original ideal, it must be a multiple of $f$, so we
add $c-mf$ our ideal to obtain

$$\{af^2+bf+c, c-mf\}$$

Step 2: A primary decomposition of this ideal gives two primary
ideals, one of which is $(f,c)$.  Since $f$ can't be zero, we throw it
away and continue with the other primary ideal:

$$(af+b+m, ac+bm+m^2, c-mf)$$

Step 3: Back to the quotient calculation.  Now our system is

$$\{af+b+m, ac+bm+m^2, c-mf, f\}$$

and we compute the Gr\"obner basis $\{f, c, b+m\}$.  This implies that
$b+m$ must also be a multiple of $f$, so we add $b+m-nf$ to our ideal,
obtaining

$$(af+b+m, ac+bm+m^2, c-mf, b+m-nf)$$

Step 4: Another primary decomposition gives another extraneous ideal
$(f,c,b+m)$.  Throwing this away, we have

$$(a+n, fn-b-m, fm-c, bm+m^2-cn)$$

Step 5: A final quotient calculation, with the system

$\{a+n, fn-b-m, fm-c, bm+m^2-cn, f\}$

gives $(f,c,b+m,a+n)$, of which the only new element, $a+n$, reduces
to zero.

So we've stabilized on 

$$(a+n, fn-b-m, fm-c, bm+m^2-cn)$$

This ideal encodes all of the information I was able to extract in
{\tt http://mathoverflow.net/questions/216392}

\vfill\eject

\section*{Bibliography}

Bronstein, Symbolic Integration I: Transcendental Functions, Springer 2004.

\begin{quote}
The single most important reference on the use of differential
algebra to solve differential equations.  The concept of
``special'' and ``normal'' polynomials is from this text.
\end{quote}

Abramov, Petkov\v sek,
On Polynomial Solutions of Linear Partial
Differential and (q-)Difference Equations.  CASC 2012.

\begin{quote}
Proves the impossibility of algorithmically solving PDEs in a polynomial
ring as a straightforward consequence of the undecidability
of Hilbert's tenth problem.
\end{quote}

Alin Bostan, Thomas Cluzeau, Bruno Salvy.
Fast Algorithms for Polynomial Solutions
of Linear Differential Equations. 2005.

\begin{quote}
{\tt http://specfun.inria.fr/bostan/publications/BoClSa05.pdf}

Explains the construction of the indicial polynomial and its
use in solving linear ODEs with polynomial coefficients.
\end{quote}

Saito, Sturmfels, Takayama, Groebner Deformations of Hypergeometric Equations.
Springer Verlag. 1999.

\begin{quote}
A standard reference work on D-modules.  Haven't read it because it's
too expensive.
\end{quote}

Harrison Tsai.  Algorithms for Algebraic Analysis.  Ph.D. thesis, UC Berkeley, 2000.

\begin{quote}
One of Sturmfels' students.  The D-module algorithms developed here
and in the previous reference require the module to be of finite
holonomic rank, which is generally not the case in the equations
of interest to us.
\end{quote}

\end{document}
